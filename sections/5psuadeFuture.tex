\subsection{The Future of PSUADE}
The next step is to implement PSUADE so that it can work in parallel with other simulations and modules. We also want PSUADE to use tuple communications instead of file I/O for its work.

Fortunately, it appears we can do both at once. 

\begin{verbatim}
I don't know whether this will useful : there is a psuade option called
'driver = psLocal' which will call a resident function inside psuade called
psLocalFunction. If you have a fixed function, you can compile it in psuade.
The psLocalFunction is in Src/DataIO/FunctionInterface.cpp.

If you have a fixed function that you call, you can put it
inside the psLocalFunction subroutine in Src/DataIO/FunctionInterface.cpp (that is,
replace everything inside that function with your code).
If you want to use your code every time a simulation is requested, you will
instantiate FunctionInterface and then call loadFunctionData

loadFunctionData(length, names)

and you set length = 5
and 
names is a char**
with names[0] = 'PSUADE_LOCAL';
        names[1] = 'NONE'
        names[2] = 'NONE'
        names[3] = 'NONE'
        names[4] = 'NONE'

Then whenever you call FunctionInterface->evaluate, it will
call your local function.
\end{verbatim}

It should be possible to substitute in the IEL\_tget function as the 'fixed function' that will be called. Then PSUADE can treat other modules as fitness functions.


The next step will be to replace PSUADE's I/O function with IEL\_tput. This shoudln't be too difficult.

\begin{verbatim}
psuade_v1.7.2/Src/DataIO
\end{verbatim}


For the tput and tget changes to PSUADE, it should follow a similar process to the section in this document describing my changes to the main function.
