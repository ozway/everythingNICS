The RC module is fairly straightforward.

\verbatiminput{extraFiles/RC.c}

Mostly what it does is set up the necessary environment, then calls the function Rf\_mainloop(). The IEL configuration file will have to have the information about where the R script is. The first argument in the args section of the configuration file must be the relative path to the R script that you are interested in running.

The RC module currently doesn't use any tuple communication, and does not interact with other modules in any way. Fortunately, this should be easily remedied. R can use the .C() function to call just about any C function. Converting C datatypes to R datatypes should be somewhat of a challenge, but there shouldn't be many challenges beyond that. It should be a very direct process to compile any C libraries into the R library, or compile them as a separate library to be loaded into R.
