The sphere packing code is fairly complicated, and I don't know how well I'll be able to explain it here. The original explanation was after months of work by 6 people in a 40 minute powerpoint with a Q\&A session afterwards, and it still didn't fully land. So with that in mind, here's the best way I can start to explain the sphere packing code.

The code consists of two major parts. One part is the genetic algorithm, which doesn't need much introduction. If you know genetic algorithms, you know this code. In short, a population of inputs is made, where each input is treated like a living organism. Its fitness is judged (by a given function), and the more fit individuals can mate with others to generate offspring that are a combination of the parents (plus some mutation). The whole population undergoes selection pressure, by which the weaker individuals are removd from the population while the more fit individuals continue to live on and reproduce.

The sphere packing code was originally a matlab code that we converted to C code. It is an attempt to address the question of hexagonal sphere packings on cylinders.
Star with a sphere of radius R, and two integers P and Q. We want to create two strings of unit spheres, one of P unit spheres and one of Q unit spheres (imagine a string of pearls). We then want to wrap those two strings of spheres around the cylinder such that the final two spheres in the chain have the same center, but all spheres except the last one are treated as solid objects. By playing with the radius of the sphere and the height difference between any two spheres in the second chain, we can solve for all of the other variables necessary to calculate the distance between the centers of the final two spheres. When that distance is zero (or sufficiently close to zero), our packing is complete.

The code's goal is to work towards the idea that for any P and Q, we can find a radius and height difference (called R and z) to complete the sphere packing. Of course, no amount of running the code will suffice as a proof, but it could find a counterexample that we did not anticipate.
